% Sample LaTeX content for offline testing
\documentclass{article}
\begin{document}

\begin{definition}\label{def:metric-space}
A \textbf{metric space} is a pair $(X,d)$ where $X$ is a set and $d : X \times X \to \mathbb{R}_{\ge 0}$ satisfies symmetry, identity of indiscernibles, and the triangle inequality.
\end{definition}

\begin{theorem}\label{thm:complete}
Every Cauchy sequence in a complete metric space converges to a point of that space.
\end{theorem}

\begin{lemma}\label{lem:cauchy-bound}
Let $(x_n)$ be a Cauchy sequence in a metric space $(X,d)$. For every $\varepsilon > 0$ there exists $N$ such that $d(x_m, x_n) < \varepsilon$ whenever $m, n > N$.
\end{lemma}

\begin{proof}
Fix $\varepsilon > 0$. Since $(x_n)$ is Cauchy there exists $N$ such that for all $m,n > N$ we have $d(x_m, x_n) < \varepsilon$. This proves the claim.
\end{proof}

\begin{corollary}\label{cor:unique-limit}
Limits in Hausdorff spaces are unique.
\end{corollary}

\end{document}
